\documentclass[10pt,twocolumn,oneside]{article}
\setlength{\columnsep}{18pt}                    %兩欄模式的間距
\setlength{\columnseprule}{0pt}                 %兩欄模式間格線粗細

\usepackage{amsthm}                             %定義,例題
\usepackage{amssymb}
\usepackage{fontspec}                           %設定字體
\usepackage{color}
\usepackage[x11names]{xcolor}
\usepackage{listings}                           %顯示code用的
\usepackage{fancyhdr}                           %設定頁首頁尾
\usepackage{graphicx}                           %Graphic
\usepackage{enumerate}
\usepackage{titlesec}
\usepackage{amsmath}
\usepackage[CheckSingle, CJKmath]{xeCJK}
\usepackage{CJKulem}

\usepackage{amsmath, courier, listings, fancyhdr, graphicx}
\topmargin=0pt
\headsep=5pt
\textheight=740pt
\footskip=0pt
\voffset=-50pt
\textwidth=545pt
\marginparsep=0pt
\marginparwidth=0pt
\marginparpush=0pt
\oddsidemargin=0pt
\evensidemargin=0pt
\hoffset=-42pt

%\renewcommand\listfigurename{圖目錄}
%\renewcommand\listtablename{表目錄}

%%%%%%%%%%%%%%%%%%%%%%%%%%%%%

\setmainfont[
    AutoFakeSlant,
    BoldItalicFeatures={FakeSlant},
    UprightFont={* Medium},
    BoldFont={* Bold}
]{Inconsolata}
%\setmonofont{Ubuntu Mono}
\setmonofont[
    AutoFakeSlant,
    BoldItalicFeatures={FakeSlant},
    UprightFont={* Medium},
    BoldFont={* Bold}
]{Inconsolata}
\setCJKmainfont{Noto Sans CJK TC}
\XeTeXlinebreaklocale "zh"                      %中文自動換行
\XeTeXlinebreakskip = 0pt plus 1pt              %設定段落之間的距離
\setcounter{secnumdepth}{3}                     %目錄顯示第三層

%%%%%%%%%%%%%%%%%%%%%%%%%%%%%
\makeatletter
\lst@CCPutMacro\lst@ProcessOther {"2D}{\lst@ttfamily{-{}}{-{}}}
\@empty\z@\@empty
\makeatother
\lstset{                                        % Code顯示
    language=C++,                               % the language of the code
    basicstyle=\footnotesize\ttfamily,          % the size of the fonts that are used for the code
    numbers=left,                               % where to put the line-numbers
    numberstyle=\scriptsize,                    % the size of the fonts that are used for the line-numbers
    stepnumber=1,                               % the step between two line-numbers. If it's 1, each line  will be numbered
    numbersep=5pt,                              % how far the line-numbers are from the code
    backgroundcolor=\color{white},              % choose the background color. You must add \usepackage{color}
    showspaces=false,                           % show spaces adding particular underscores
    showstringspaces=false,                     % underline spaces within strings
    showtabs=false,                             % show tabs within strings adding particular underscores
    frame=false,                                % adds a frame around the code
    tabsize=2,                                  % sets default tabsize to 2 spaces
    captionpos=b,                               % sets the caption-position to bottom
    breaklines=true,                            % sets automatic line breaking
    breakatwhitespace=true,                     % sets if automatic breaks should only happen at whitespace
    escapeinside={\%*}{*)},                     % if you want to add a comment within your code
    morekeywords={*},                           % if you want to add more keywords to the set
    keywordstyle=\bfseries\color{Blue1},
    commentstyle=\itshape\color{Red1},
    stringstyle=\itshape\color{Green4},
}


\begin{document}
\pagestyle{fancy}
\fancyfoot{}
%\fancyfoot[R]{\includegraphics[width=20pt]{ironwood.jpg}}
\fancyhead[C]{Ali88}
\fancyhead[R]{\thepage}
\renewcommand{\headrulewidth}{0.4pt}
\renewcommand{\contentsname}{Contents}

\scriptsize
\tableofcontents
\section{語法}
    \subsection{c++}
        \lstinputlisting{Contents/section1/basic.cpp}
    \subsection{python}
        \lstinputlisting{Contents/section1/tmp.py}

\section{Graph}
    \subsection{Bellman-Ford}
        \lstinputlisting{Contents/Graph/Bellman-Ford.cpp}
    \subsection{Dijkstra}
        \lstinputlisting{Contents/Graph/Dijkstra.cpp}
    \subsection{Floyd-Warshall}
        \lstinputlisting{Contents/Graph/Floyd-Warshall.cpp}
    \subsection{SPFA}
        \lstinputlisting{Contents/Graph/SPFA.cpp}
    \subsection{smallTree}
        \lstinputlisting{Contents/Graph/tree.cpp}

\section{Other}
    \subsection{KM}
        \lstinputlisting{Contents/Other/KM.cpp}
    \subsection{LCS}
        \lstinputlisting{Contents/Other/LCS.cpp}
    \subsection{LIS}
        \lstinputlisting{Contents/Other/LIS.cpp}
    \subsection{merge}
        \lstinputlisting{Contents/Other/merge.cpp}  
    \subsection{Prime}
        \lstinputlisting{Contents/Other/Prime.cpp}
    \subsection{UVA12321}
        \lstinputlisting{Contents/Other/UVA12321.cpp}
    \subsection{Fire}
        \lstinputlisting{Contents/Other/Fire.cpp}
    \subsection{ALLSUM}
        \lstinputlisting{Contents/Other/ALLSUM.cpp}

\section{ENDLN}
    \subsection{Bipatirate}
        \lstinputlisting{Contents/ENDLN/Bipatirate.cpp}
    \subsection{LCA}
        \lstinputlisting{Contents/ENDLN/LCA.cpp}
    \subsection{Trie}
        \lstinputlisting{Contents/ENDLN/Trie.cpp}
    \subsection{GCD&LCM}
        \lstinputlisting{Contents/ENDLN/GCD_LCM.cpp}
    


\end{document}
